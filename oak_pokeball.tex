\documentclass[a4paper,12pt]{jsarticle}
\usepackage{graphicx}
\usepackage{geometry}
\usepackage{amssymb}
\usepackage{amsmath}

% Page layout settings
\geometry{left=25mm, right=25mm, top=30mm, bottom=30mm}

\begin{document}
$n$枚の山札のうち, $m$枚がたねポケモンであるとする. $m$枚のたねポケモンのうち, $1$枚が当たりである.
オーキド博士とモンスターボールを使って当たりを引く確率を求める.(便宜上、$3 \leqq m \leqq n-2$とする.)

オーキド→ボールと使った場合, オーキドで引くたねの枚数ごとに場合分けすることで
\begin{equation*}
    p_1 = \frac{m(m-1)}{n(n-1)} \cdot \frac{3}{m}
    + \frac{2m(n-m)}{n(n-1)} \cdot \frac{2}{m}
    + \frac{2(n-m)(n-m-1)}{n(n-1)} \cdot \frac{1}{m}
    = \frac{m^2+2n^2-m-2n}{mn(n-1)}
\end{equation*}

ボール→オーキドと使った場合
\begin{equation*}
    p_2 = \frac{1}{m} + \left(1-\frac{1}{m}\right) \cdot \frac{2}{n-1} = \frac{2m+n-3}{m(n-1)}
\end{equation*}

その差は
\begin{equation*}
    p_1 - p_2 = \frac{m^2-2mn+n^2-m+n}{mn(n-1)} = \frac{(n-m)(n-m-1)}{mn(n-1)} > 0
\end{equation*}
より, 常にオーキド→ボールと使った方が当たりを引く確率が高い.

例えば1ターン目を想定して$n=14, m=3$の場合を考えると,

オーキド→ボールと使った場合67.8%で狙いのカードを引けるのに対して,

ボール→オーキドと使った場合43.5%でしか引くことができない.

人事を尽くして天命を待つべし. 先1フリーザー?そんなものは知らん.
\end{document}